\chapter{Multi-instanční učení}
Multi-instanční učení (anglicky \textenglish{\textit{multi-instance learning, MIL}}), poprvé takto popsané v \cite{dietterich_solving_1997}, je variací metody učení s učitelem (anglicky \textenglish{\textit{supervised learning}}), metody pro určení klasifikační funkce z trénovacích dat. Nechť existují vstupní objekty ze vstupního prostoru \( \BPspace X \), výstupní objekty z výstupního prostoru \( \BPspace Y \), kde každému \( x \in \BPspace X \) lze jednoznačně přiřadit \( y \in \BPspace Y \). Tyto výstupní objekty avšak nejsou známé ani pro trénovací data. Proto jsou vstupní objekty seskupeny do takzvaných tašek (anglicky \textenglish{\textit{bag}}).
\begin{define}
	Nechť \( \BPspace X \) je prostorem vstupních objektů. Potom prostorem tašek je množina \( \BPspace B \) splňující podmínky
	\begin{enumerate}
		\item \( \BPspace B \subset \mathcal P \left( \BPspace X \right) \)
		\item \( \left( \forall x \in \BPspace X \right) \left( \exists ! b \in \BPspace B \right) \left( x \in b \right) \)
	\end{enumerate}
	Prvky prostoru tašek se nazývají taškami.
\end{define}
\begin{define}\label{baglabel}
	Nechť \( \BPspace X \) je prostorem vstupních objektů, \( \BPspace Y \) je prostorem výstupních objektů, kde platí
	\[ \left( \forall x \in \BPspace X \right) \left( \exists ! y_x \in \BPspace Y \right) \]
	Nechť \( \BPspace B \) je prostorem tašek. Nechť je v prostoru \( \BPspace Y \) definována funkce maximum. Potom výstupním objektem příslušícím tašce \( b \in \BPspace B \) je
	\[ y_b = \max_{x \in b} \left( y_x \right) \in \BPspace Y \]
\end{define}

\section{Metody řešení multi-instančních úloh}

V následující sekci jsou popsány tři převládající přístupy k řešení Multi-instančních problémů, více k problematice srov. \cite{pevny_using_2016}.

\subsection{Paradigma prostoru instancí}
Je předpokládána existence výstupních objektů pro všechny vstupní objekty (odpovídající instancím), přestože tyto hodnoty nejsou známé. Pro každou tašku je pak předpokládána exitence výstupního objektu \( y_b \). Cílem metody je najít klasifikační funkci \( f : \BPspace X \to \BPspace Y \) a posléze určit
\begin{equation}
	y_b = \max_{x \in b} \left( f \left( x \right) \right)
\end{equation}

\subsection{Paradigma prostoru tašek}
Předpoklady úlohy jsou omezeny pouze na existenci výstupních objektů na úrovni tašek. Lze definovat kernel funkci
\begin{equation}
	k : \BPspace B \times \BPspace B \to \BPfield R_0^+
\end{equation}
kterou je možno použít jako metriku při klasifikaci například algoritmem k-nejbližsích sousedů.

\subsection{Paradigma vloženého prostoru}
Předpoklady úlohy jsou (stejně jako u paradigmatu prostoru tašek) omezeny pouze na existenci výstupních objektů na úrovni tašek. Nechť existuje vkládající funkce \( \phi : \BPspace B \to \bar{\BPspace X} \), kde \( \bar{\BPspace X} \) je nějaký vstupní prostor. Díky této funkci lze reprezentovat kažkou tašku vstupním objektem \( \phi \left( b \right) \in \bar{\BPspace X} \) a nasledně lze použít jakýkoliv algoritmus používaný při klasickém učení s učitelem. Pokud \( \bar{\BPspace X} = \BPfield R^n \), pak je většinou kladena na \( \phi \) podmínka
\begin{equation}
	\left( \exists \psi : \BPfield R^k \to \BPfield R \right) \left( \phi \left( x^{(1)}, \dots, x^{(n)} \right) = \left( \psi \left( x_1^{(1)}, \dots, x_1^{(n)} \right), \dots, \psi \left( x_k^{(1)}, \dots, x_k^{(n)} \right) \right) \right)
\end{equation}

\section{Použitý formalismus}\label{used_formalism}
V následujících částech bude používáno \( \BPspace X = \BPfield R^n \), \( \BPspace Y = \left\{ -1, +1 \right\} \) a \( \BPspace B = \bigcup_{k = 1}^{\infty} \left( \BPfield R^n \right) ^k \). Prvky \( \BPspace X \) budou nazývány \textit{vektory příznaků}, prvky \( \BPspace Y \) budou nazývány \textit{značky}. Instance bude považována za \textit{pozitivní}, pokud je pozitivní její značka, a negativní jinak. Při použití definice \ref{baglabel} bude taška považována za pozitivní, pokud obsahuje alespoň jednu pozitivní instanci, a negativní, pokud obsahuje pouze negativní instance. Je použito paradigma vloženého prostoru a dále popsaný formalismus, převzatý z \cite{pevny_using_2016} a \cite{pevny_discriminative_2016}.

Nechť existuje prostor rozdělení pravděpodobnosti \( \BPspace P^{\BPspace X} \). Každá taška \( b \) je pak realizací nějaké pravděpodobnostní funkce \( P \left( p_b, y_b \right) \) kde \( p_b \in \BPspace P^{\BPspace X} \) je rozdělení pravděpodobnosti instancí v \( b \), a \( y_b \in \BPspace Y \) je značka této tašky. Je tedy předpokladem, že pravděpodobnostní funkce závisí i na značce, tj. \( p_+ \neq p_- \) kde \( p_+ \sim P(p, +1) \) a \( p_- \sim P(p, -1) \). Prostor všech tašek lze poté zapsat jako
\begin{equation}
	\BPspace B = \left\{ \left\{ x_1, x_2, \dots, x_n \right\} \in \BPspace P \left( \BPspace X \right) \middle| \left( \forall i \in \hat n \right) \left( x_i \sim p \in \BPspace P^{\BPspace X} \right) \right\}
\end{equation}

Pokud jsou zvoleny funkce \( k : \BPspace X \to \BPspace X \) a \( g : \bigcup_{k = 1}^{\infty} \BPspace X^k \to \bar{\BPspace X} \), lze vkládající funkci definovat jako
\begin{equation}\label{net_pooling}
	\phi : \BPspace B \to \bar{\BPspace X} : b \mapsto g \left( \left\{ k \left( x \right) \middle| x \in b \right\} \right)
\end{equation}
Za \( g \) lze volit například funkce minimum, maximum, aritmetický průměr. Funkce \( k \) je potom určována algoritmem pro učení s učitelem. \todo{Obrázek z \cite{pevny_using_2016}}\todo{Zapsat lépe}Tedy neuronové síťě určující funkce \( f \) a \( k \) jsou trénovány najednou.
